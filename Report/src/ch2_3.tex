\section{METATRON}
    Το άρθρο "\textit{MetaTron: advancing biomedical annotation empowering relation annotation and collaboration}" των Irrera, Marchesin κ.α. πραγματεύεται το εργαλείο MetaTron.

    Πρόκειται για μια εφαρμογή ανοιχτού κώδικα με σκοπό τη βελτίωση της αποτελεσματικότητας του σχολιασμού (annotation) βιοϊατρικών δεδομένων. \cite{MetaTron}

    \subsection{Τι είναι ο σχολιασμός (annotating);}
        Πρόκειται για την προσθήκη δομημένης πληροφορίας σε βιοϊατρικά κείμενα με σκοπό τη βελτίωσης της χρηστικότητάς τους για ερευνητικούς σκοπούς ή για κλινικές εφαρμογές.

        Συνήθως περιλαμβάνει τον εντοπισμό και την επισήμανση γονιδίων, πρωτεϊνών και άλλων οντοτήτων, τον καθορισμό σχέσεων μεταξύ τους (πώς ένα φάρμακο αλληλεπιδρά με μια ασθένεια ή πως σχετίζονται δύο γονίδια μεταξύ τους).
        Για να επιτευχθεί, χρησιμοποιούνται οντολογίες όπου βοηθούν στον σαφή καθορισμό των σχέσεων των εννοιών, μετατρέποντας τις έννοιες σε machine-readable οδηγώντας στον αυτοματισμό και στην εξόρυξη γνώσης.

        \subsubsection{ΠΡΟΕΚΤΑΣΗ: Κριτήρια (χειροκίνητου) σχολιασμού}
            Ο χειροκίνητος σχολιασμός, δηλαδή ο σχολιασμός που προέρχεται από τους ίδιους τους επιστήμονες \footnote{Ο αυτοματοποιημένος σχολιασμός είναι μια διαδικασία που στηρίζεται σε αλγορίθμους και στη μηχανική μάθηση.} είναι μια διαδικασία κουραστική και χρονοβόρα.
            Απαιτεί εξαιρετική εξειδίκευση από τους επιστήμονες ώστε να μπορέσουν να ταξινομήσουν με ακρίβεια τις οντότητες, να ελέγξουν για λάθη, κάτι το οποίο κοστίζει όλο και περισσότερο όσο αυξάνεται η πολυπλοκότητα και το μέγεθος των δεδομένων.

            Υπάρχουν πάρα πολλά κριτήρια που επιζητούνται από τα λογισμικά που χρησιμοποιούνται για χειροκίνητο σχολιασμό, που αφορούν τα τεχνικά χαρακτηριστικά τους, τη χρηστικότητά τους, κ.α.
            Παραδείγματα για τα τεχνικά χαρακτηριστικά είναι η διαθεσιμότητα του κώδικα, η ευκολία εγκατάστασης, η ποιότητα του documentation, το κόστος, ενώ για τη χρηστικότητά τους οι σημειώσεις πολλαπλών ετικετών (multi-label annotations), ενσωμάτωση με οντολογίες, προσχολιασμούς βάση προηγούμενων δεδομένων, απόρρητο δεδομένων κα.

            Έρευνα \cite{ManualAnnotating} που έγινε για τα σημαντικότερα χαρακτηριστικά, κατέληξε ότι τα σημαντικότερα είναι:
            \vspace{-10pt}
            \begin{itemize}[label={\tiny \blacksquare}]
                \item να είναι διαθέσιμο διαδικτυακά ή να μπορεί να εγκατασταθεί εύκολα
                \vspace{-7pt}
                \item να είναι λειτουργικό, με διαισθητικές λειτουργίες χωρίς να απαιτεί μεγάλο επίπεδο εμπειρίας από τον χρήστη
                \vspace{-7pt}
                \item να υποστηρίζει σχηματική αναπαράσταση με ευέλικτα εργαλεία που να καλύπτουν πολλά διαφορετικά use-cases
            \end{itemize}
            \vspace{-10pt}

            Τα λογισμικά σχολιασμού μέχρι τώρα (και αυτά που βασίζονται στη βιοϊατρική\footnote{MedTAG, BioQRator, ezTag, MyMiner, TeamTAT κα.}, και γενικού σκοπού\footnote{DocTag, brat, Catma, FLAT, LightTag, PDFAnno κα.}) δεν κάλυπταν όλα αυτά τα κριτήρια και χαρακτηριστικά ταυτόχρονα, με κάποια να υπερτερούν σε μερικά και να υστερούν σε άλλα.
            Επομένως, είναι σαφής η ανάγκη για έναν πιο αποτελεσματικό, λογισμικό σχολιασμού.


    \subsection{Ο ρόλος του Metatron}
        Το Metatron είναι από τα λίγα εργαλεία σχολιασμού που καταφέρνει να είναι αποτελεσματικό σε όλα τα προαναφερόμενα χαρακτηριστικά.
        Υποστηρίζει διαφορετικούς τύπους αρχείων, μπορεί να συνδεθεί με APIs από πήγες όπως PubMed για πρόσβαση σε επιπλέον πηγές, περιλαμβάνει διαφορετικούς τύπους σχολιασμού, δυνατότητες για συνεργατικό σχολιασμό, αυτόματες προτάσεις, παραμετροποίηση κα.

    \subsection{Χαρακτηριστικά του MetaTron}
        Το MetaTron υποστηρίζονται πολλαπλοί τύποι σχολιασμού, σχολιασμός σε επίπεδο εγγράφου (document-level) που περιλαμβάνουν την ανάθεση ετικετών σε ολόκληρα έγγραφα, και σε επίπεδο αναφοράς (mention-level) που εστιάζουν σε συγκεκριμένα τμήματα του κειμένου.
        Ο σχολιασμός σε επίπεδο κειμένου περιλαμβάνει σχόλια-ετικέτες (labels) και σχόλια-ισχυρισμούς (assertions), τα οποία μπορούν να συμπεριληφθούν σε RDF γραφήματα για καλύτερη αναπαράσταση.

        Υποστηρίζονται οι οντολογίες, επιτρέποντας στους χρήστες να ορίζουν εννοιών (concepts\footnote{Το MetaTron ορίζει ως έννοια ένα μεμονωμένο, αναγνωρίσιμο αντικείμενο με ξεχωριστή και ανεξάρτητη ύπαρξη.}),
            ο συνεργατικός σχολιασμός, πολλαπλοί τύποι αρχείων και μεγάλη παραμετροποίηση.
        Επιπλέον, περιλαμβάνει το AutoTron, ένα χαρακτηριστικό που προσφέρει προβλέψεις στο σύστημα για τον αυτοματοποιημένο σχολιασμό, σκοπεύοντας στην ενίσχυση της αποτελεσματικότητας των χρηστών.

        \subsubsection{Αρχιτεκτονική του MegaTron}
        Η αρχιτεκτονική του MegaTron χωρίζεται σε τρία επίπεδα, το επίπεδο δεδομένων (data layer), το επιχειρησιακό επίπεδο (business layer) και το επίπεδο παρουσίασης (presentation layer).

        Το επιχειρησιακό επίπεδο χρησιμοποιεί ένα REST API σε Django Python, δρώντας ως ο μεσολαβητής ανάμεσα στο επίπεδο παρουσίασης και επίπεδο δεδομένων.
        Το επίπεδο παρουσίασης αναπτύχθηκε χρησιμοποιώντας ReactJS, HTML/CSS/JS.

    \subsection{Υλοποίηση και αποτελέσματα}
        Το άρθρο μπαίνει σε μια λεπτομερή περιγραφή των χαρακτηριστικών και του τρόπου λειτουργίας του λογισμικού, όπως επίσης και έρευνα των χρηστών για το πόσο έμειναν ικανοποιημένοι, πράγματα που ανήκουν εκτός της σφαίρας της μελέτης μας.
