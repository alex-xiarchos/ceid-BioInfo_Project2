\chapter{ΕΙΣΑΓΩΓΗ}

    Η βιοπληροφορική έχει αναδειχθεί σαν ένα κομβικό επιστημονικό πεδίο ανάμεσα στη βιολογία και την επιστήμη των υπολογιστών.
    Χρησιμοποιεί υπολογιστικά εργαλεία για τη μελέτη και την κατανόηση βιολογικών δεδομένων όπως το DNA και οι πρωτεϊνες, μια διαδικασία που η ραγδαία πρόοδος των επιστημών κατέστησε απαραίτητη.
    Η βιοπληροφορική πλέον έχει εξελιχθεί σε ένα αναγκαίο εργαλείο για ερευνητικούς σκοπόυς, στην ανακάλυψη νέων φαρμάκων, στην εξατομικευμένη και προληπτική ιατρική, στη γονιδιακή θεραπεία, στη βελτιώση της καλλιέργειας κ.α.
    Η χρήση της βοηθάει στην επέκταση της γνώσης πολύ πιο αποτελεσματικά και με μεγαλύτερη ακρίβεια.

    Ο συγκεκριμένος τομέας ήρθε στο προσκήνιο με την ανακάλυψη του ανθρώπινου γονιδιώματος, κάτι που οι παραδοσιακές μέθοδοι ανάλυσης δεδομένων ήταν ανεπαρκείς για να χειριστούν τον τεράστιο όγκο των πληροφοριών που παραγόνταν.

    Μια μεγάλη πρόκληση στη βιοπληροφορική είναι η ανάγκη για τυποποίηση των τρόπων αναπαράστασης και ανταλλαγής πολύπλοκων βιολογικών δεδομένων.
    Εδώ είναι που η XML (eXtensive Markup Language) μπαίνει στο προσκήνιο.
    Πρόκειται για μια γλώσσα σήμανσης αρκετά ευέλικτη και ισχυρή για την αναπαράσταση ιεραρχικών σχέσεων (hierarchical relationships), κάτι αρκετά κοινότυπο στη μελέτη βιολογικών δεδομένων.

    Η σύνδεση μεταξύ XML και βιοπληροφορικής είναι τόσο διαδεδομένη που έχει οδηγήσει στην ανάπτυξη διάφορων ευρέως χρησιμοποιούμενων προτύπων στο τομέα που θα αναλυθούν στη συνέχεια.
    Καθώς ο τομέας της βιοπληροφορικής συνεχίζεται να εξελίσσεται, ο ρόλος των γλωσσών σήμανσης όπως η XML και των προτύπων που αυτή δημιουργεί γίνεται ολοένα και πιο σημαντικός για την ανταλλαγή και ενοποίηση δεδομένων και την παραγωγικότητα της επιστημονικής έρευνας.


% Να αναφέρω κάποιες XML υλοποιήσεις:

%The connection between XML and bioinformatics is exemplified by several widely-used data formats and standards in the field.
%    For instance, the Systems Biology Markup Language (SBML) is an XML-based format used to represent models of biological processes.
%    Similarly, the Protein Data Bank (PDB) format, which describes three-dimensional structures of proteins and nucleic acids, has an XML version called PDBML.
%    These XML-based formats enable researchers to exchange complex biological data in a structured and machine-readable manner, facilitating collaboration and data integration across different platforms and tools.
%Moreover, XML's extensibility allows for the creation of custom tags and attributes, making it possible to adapt the format to the specific needs of different subfields within bioinformatics.
%    This flexibility has led to the development of numerous XML-based standards for various aspects of biological data representation, such as MAGE-ML for microarray gene expression data, PhyloXML for phylogenetic trees, and BIOML for general biological data exchange.