\chapter{ΕΙΣΑΓΩΓΗ}

    Η βιοπληροφορική έχει αναδειχθεί σαν ένα κομβικό επιστημονικό πεδίο ανάμεσα στη βιολογία και την επιστήμη των υπολογιστών.
    Χρησιμοποιεί υπολογιστικά εργαλεία για τη μελέτη και την κατανόηση βιολογικών δεδομένων όπως το DNA και οι πρωτεϊνες, μια διαδικασία που η ραγδαία πρόοδος των επιστημών κατέστησε απαραίτητη.
    Η βιοπληροφορική πλέον έχει εξελιχθεί σε ένα αναγκαίο εργαλείο για ερευνητικούς σκοπόυς, στην ανακάλυψη νέων φαρμάκων, στην εξατομικευμένη και προληπτική ιατρική, στη γονιδιακή θεραπεία, στη βελτιώση της καλλιέργειας κ.α.
    Η χρήση της βοηθάει στην επέκταση της γνώσης πολύ πιο αποτελεσματικά και με μεγαλύτερη ακρίβεια.

    Ο συγκεκριμένος τομέας ήρθε στο προσκήνιο με την ανακάλυψη του ανθρώπινου γονιδιώματος, κάτι που οι παραδοσιακές μέθοδοι ανάλυσης δεδομένων ήταν ανεπαρκείς για να χειριστούν τον τεράστιο όγκο των πληροφοριών που παραγόνταν.

    Μια μεγάλη πρόκληση στη βιοπληροφορική είναι η ανάγκη για τυποποίηση των τρόπων αναπαράστασης και ανταλλαγής πολύπλοκων βιολογικών δεδομένων.
    Εδώ είναι που η XML (eXtensive Markup Language) μπαίνει στο προσκήνιο.
    Πρόκειται για μια γλώσσα σήμανσης αρκετά ευέλικτη και ισχυρή για την αναπαράσταση ιεραρχικών σχέσεων (hierarchical relationships), κάτι αρκετά κοινότυπο στη μελέτη βιολογικών δεδομένων.

    Η σύνδεση μεταξύ XML και βιοπληροφορικής είναι τόσο διαδεδομένη που έχει οδηγήσει στην ανάπτυξη διάφορων ευρέως χρησιμοποιούμενων προτύπων στο τομέα:


\chapter{ΑΝΑΣΚΟΠΗΣΗ ΤΩΝ ΑΡΘΡΩΝ}
    
    \section{ΑΡΘΡΟ 1}
