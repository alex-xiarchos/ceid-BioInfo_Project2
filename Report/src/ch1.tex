\chapter{ΕΙΣΑΓΩΓΗ}

    Η βιοπληροφορική έχει αναδειχθεί σαν ένα κομβικό επιστημονικό πεδίο ανάμεσα στη βιολογία και την επιστήμη των υπολογιστών.
    Χρησιμοποιεί υπολογιστικά εργαλεία για τη μελέτη και την κατανόηση βιολογικών δεδομένων όπως το DNA και οι πρωτεΐνες, μια διαδικασία που η ραγδαία πρόοδος των επιστημών κατέστησε απαραίτητη.
    Ο συγκεκριμένος τομέας ήρθε στο προσκήνιο με την ανακάλυψη του ανθρώπινου γονιδιώματος, κάτι που οι παραδοσιακές μέθοδοι ανάλυσης δεδομένων ήταν ανεπαρκείς για να χειριστούν τον τεράστιο όγκο των πληροφοριών που παραγόνταν.

    Η βιοπληροφορική πλέον έχει εξελιχθεί σε ένα αναγκαίο εργαλείο για ερευνητικούς σκοπόυς, στην ανακάλυψη νέων φαρμάκων, στην εξατομικευμένη και προληπτική ιατρική, στη γονιδιακή θεραπεία, στη βελτιώση της καλλιέργειας κ.α.,
        οδηγώντας σε συμπεράσματα πολύ πιο αποτελεσματικά και με μεγαλύτερη ακρίβεια.

    Μια μεγάλη πρόκληση στη βιοπληροφορική είναι η ανάγκη για τυποποίηση των τρόπων αναπαράστασης και ανταλλαγής πολύπλοκων βιολογικών δεδομένων.
    Εδώ είναι που η XML (eXtensive Markup Language) μπαίνει στο προσκήνιο.
    Πρόκειται για μια γλώσσα σήμανσης αρκετά ευέλικτη και ισχυρή για την αναπαράσταση ιεραρχικών σχέσεων (hierarchical relationships), κάτι αρκετά κοινότυπο στη μελέτη βιολογικών δεδομένων.
    Διαδραματίζει πολύ σημαντικό ρόλο στη μεταφορά βιολογικής πληροφορίας μεταξύ συστημάτων, λογισμικών και βάσεων δεδομένων.
    Μάλιστα η σύνδεση μεταξύ XML και βιοπληροφορικής είναι τόσο διαδεδομένη που έχει οδηγήσει στην ανάπτυξη διάφορων ευρέως χρησιμοποιούμενων προτύπων στο τομέα που θα αναλυθούν στη συνέχεια.

    Καθώς ο τομέας της βιοπληροφορικής συνεχίζεται να εξελίσσεται, ο ρόλος των γλωσσών σήμανσης όπως η XML και των προτύπων που αυτή δημιουργεί γίνεται ολοένα και πιο σημαντικός για την ανταλλαγή και ενοποίηση δεδομένων και την παραγωγικότητα της επιστημονικής έρευνας,
        έχοντας δημιουργήσει αρκετές γλώσσες που βασίζονται στην XML και που παίζουν καίριο ρόλο στη μεταφορά δεδομένων βιολογικής φύσεως.
    Παραδείγματα τέτοιων γλωσσών είναι η SBML (Systems Biology Markup Language), PDMBL (Protein Data Bank Markup Language) και άλλες.

    Στην συγκεκριμένη εργασία θα δοθεί βάση στους τρόπους αναπαράστασης βιολογικής πληροφορίας με την χρήση XML-based γλωσσών, τα συστήματα που γίνεται αποθήκευση και μεταφορά αυτής της πληροφορίας, ο τρόπος που είναι δομημένη, πώς μπορεί να επιτευχθεί ομογένεια λόγω των διαφορετικών πηγών πληροφορίας που οδηγεί σε αρχεία διαφορετικού τύπου μορφολογίας, κ.α.